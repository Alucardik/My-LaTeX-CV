%%%%%%%%%%%%%%%%%
% This is an sample CV template created using altacv.cls
% (v1.1.4, 27 July 2018) written by LianTze Lim (liantze@gmail.com). Now compiles with pdfLaTeX, XeLaTeX and LuaLaTeX.
% 
%% It may be distributed and/or modified under the
%% conditions of the LaTeX Project Public License, either version 1.3
%% of this license or (at your option) any later version.
%% The latest version of this license is in
%%    http://www.latex-project.org/lppl.txt
%% and version 1.3 or later is part of all distributions of LaTeX
%% version 2003/12/01 or later.
%%%%%%%%%%%%%%%%

%% If you need to pass whatever options to xcolor
\PassOptionsToPackage{dvipsnames}{xcolor}

%% If you are using \orcid or academicons
%% icons, make sure you have the academicons 
%% option here, and compile with XeLaTeX
%% or LuaLaTeX.
% \documentclass[10pt,a4paper,academicons]{altacv}

%% Use the "normalphoto" option if you want a normal photo instead of cropped to a circle
% \documentclass[10pt,a4paper,normalphoto]{altacv}

\documentclass[10pt,a4paper]{altacv}
%% AltaCV uses the fontawesome and academicon fonts
%% and packages. 
%% See texdoc.net/pkg/fontawecome and http://texdoc.net/pkg/academicons for full list of symbols.
%% 
%% Compile with LuaLaTeX for best results. If you
%% want to use XeLaTeX, you may need to install
%% Academicons.ttf in your operating system's font 
%% folder.


% Change the page layout if you need to
\geometry{left=1cm,right=9cm,marginparwidth=6.8cm,marginparsep=1.2cm,top=1.25cm,bottom=1.25cm,footskip=2\baselineskip}

% Change the font if you want to.

% If using pdflatex:
\usepackage[T1]{fontenc}
\usepackage[utf8]{inputenc}
\usepackage[default]{lato}

% If using xelatex or lualatex:
% \setmainfont{Lato}

% Change the colours if you want to
\definecolor{Mulberry}{HTML}{72243D}
\definecolor{SlateGrey}{HTML}{2E2E2E}
\definecolor{LightGrey}{HTML}{666666}
\colorlet{heading}{Sepia}
\colorlet{accent}{Mulberry}
\colorlet{emphasis}{Black}
\colorlet{body}{LightGrey}

% Change the bullets for itemize and rating marker
% for \cvskill if you want to
\renewcommand{\itemmarker}{{\small\textbullet}}
\renewcommand{\ratingmarker}{\faCircle}
%% sample.bib contains your publications
\addbibresource{sample.bib}

\usepackage[colorlinks]{hyperref}

\begin{document}

\name{CHEKIN IGOR }
\tagline{FRONTEND DEVELOPER / DATA ANALYST }

\personalinfo{%
  % Not all of these are required!
  % You can add your own with \printinfo{symbol}{detail}
  \email{iachekin@edu.hse.ru }
  \phone{+79191006680}
  \mailaddress{Moscow, Russia }
  \location{Moscow, Russia}
  \text{Telegram: @AlucardikAnno}
  %% You MUST add the academicons option to \documentclass, then compile with LuaLaTeX or XeLaTeX, if you want to use \orcid or other academicons commands.
%   \orcid{orcid.org/0000-0000-0000-0000}
}

%% Make the header extend all the way to the right, if you want. 
\begin{fullwidth}
\makecvheader
\end{fullwidth}

%% Depending on your tastes, you may want to make fonts of itemize environments slightly smaller
% \AtBeginEnvironment{itemize}{\small}


%% Provide the file name containing the sidebar contents as an optional parameter to \cvsection.
%% You can always just use \marginpar{...} if you do
%% not need to align the top of the contents to any
%% \cvsection title in the "main" bar.

\cvsection[page1sidebar]{Experience}

\cvevent{To be filled}{To be filled}{To be filled}{To be filled}
\begin{itemize}
\item To be filled
\item To be filled 
\item To be filled
\item To be filled 
\end{itemize}

\divider

\cvevent{To be filled}{To be filled}{To be filled}{To be filled}
\begin{itemize}
\item To be filled 

\item To be filled  
\end{itemize}

\vspace{7mm}
\cvsection{Projects}

\cvevent{Project 1}{Online Testing System "My Code"}{1 month}{}
\begin{itemize}
\item Created online learning programming resource prototype on "Leaders of Digital Transformation" hackathon as part of "Galamtor" team
\end{itemize}

\divider

\cvevent{Project 2}{Instagram-like service}{3 months}{}
\begin{itemize}
\item A web application to share photos between people 
\end{itemize}


\vspace{7mm}
\cvsection{Skills}

\cvtag{Data Structures}
\cvtag{Algorithms}
\cvtag{C}
\cvtag{C++}
\cvtag{C-sharp}
\cvtag{Python3}

\divider\smallskip
\cvtag{Vue.js}
\cvtag{JS}
\cvtag{HTML}
\cvtag{CSS}

\divider\smallskip
\cvtag{Git}
\cvtag{QT}
\cvtag{Linux}


\medskip

\clearpage

%% If the NEXT page doesn't start with a \cvsection but you'd
%% still like to add a sidebar, then use this command on THIS
%% page to add it. The optional argument lets you pull up the 
%% sidebar a bit so that it looks aligned with the top of the
%% main column.
% \addnextpagesidebar[-1ex]{page3sidebar}

\end{document}
